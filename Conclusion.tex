\chapter*{Conclusion et perspectives}
\addcontentsline{toc}{chapter}{Conclusion et perspectives}
\markboth{Conclusion et perspectives}{Conclusion et perspectives}
\label{sec:conclusion}

Dans ce rapport de recherche, nous avons proposé l'intégration des méthodes d'estimation non paramétriques dans le monde de l'apprentissage automatique pour optimiser et trouver le nombre optimal d'époques pour de meilleures performances et nous avons développé une métrique capable d'évaluer de manière objective la performance des classifieurs.\\
Ce chapitre résume d'abord les travaux proposés, puis une discussion sur les enjeux et contributions de la recherche est présentée, avec des pistes de travaux futurs.
\section*{Conclusion}
En conclusion, notre travail de recherche a permis de mettre en lumière l'importance des méthodes d'estimation non paramétriques dans le domaine de l'apprentissage automatique, en particulier pour l'optimisation de la valeur du nombre d'époques pour obtenir des performances de modèles supérieures. Nous avons réalisé plusieurs simulations pour évaluer la performance des différentes méthodes non paramétriques, afin de sélectionner la plus adaptée à notre problème. De plus, nous avons développé une métrique qui se base sur les densités de probabilité des variables, capable d'évaluer de manière objective la performance des classifieurs.
\setlength{\parindent}{0pt}
Nous avons également présenté un exemple pratique d'application des méthodes paramétriques dans le contexte de l'apprentissage automatique. Ce travail contribue ainsi à l'amélioration de la qualité des modèles en machine learning et peut avoir des applications dans divers domaines, tels que la reconnaissance de forme, la prédiction de séries temporelles, la classification d'images, etc. Enfin, des travaux futurs pourraient inclure l'exploration d'autres méthodes non paramétriques et l'évaluation de leur performance, ainsi que l'application de ces méthodes à d'autres problèmes en apprentissage automatique.
\section*{Perspectives}
 Des améliorations peuvent être apportées à notre travail, en particulier en ce qui concerne la gestion des données et des ressources avec beaucoup de bruit. Nous pourrions également explorer d'autres méthodes non paramétriques et évaluer leurs performances, ainsi qu'appliquer ces méthodes à d'autres problèmes en apprentissage automatique.
\setlength{\parindent}{0pt}
 Enfin, nous pourrions envisager d'intégrer des techniques hybrides de recommandation pour améliorer davantage notre système et lui permettre de fournir des résultats plus précis et personnalisés.


