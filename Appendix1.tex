\chapter{Appendix 1}
\section{Introduction}
Fuzzy set theory was introduced, in 1965, by Zadeh
\cite{fuzzysets}. It is considered as a useful theory for modeling
and reasoning with imprecise knowledge. Fuzzy set theory is a mathematical theory where the fuzziness
is the ambiguity that can be found in the definition of a
concept or the meaning of a word \cite{7f}. Imprecision in
expressions like ``low frequency", ``high demand" or ``small
number" can be called fuzziness. In this Appendix, the basics of fuzzy sets will be introduced. 

In Section A.2, the basics of fuzzy set theory will be given. In Section A.3, the main notions of the fuzzy set membership functions will be introduced while in Section A.4, the fundamental operations of fuzzy sets will be highlighted. Finally, in Section A.5, the process of fuzzy logic will be described.

\section{Conclusion}
In this Appendix, we have elucidated the basics of fuzzy set
theory which is a generalization of the classical set theory.
Offering a natural model,  fuzzy sets are used to handle imprecise information.

